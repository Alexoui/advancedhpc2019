\documentclass{article} \usepackage[utf8]{inputenc} \title{Report.4.investigate} \author{alexandre caro } 
\date{October 2019} \begin{document} \maketitle \begin{verbatim} We implement those change to use 2D: __global__ 
void grayscale2(uchar3 *input, uchar3 *output) { int tx = threadIdx.x + blockIdx.x * blockDim.x; int ty = 
threadIdx.y + blockIdx.y * blockDim.y; int w = blockDim.x * gridDim.x; int tid = ty*w + tx; output[tid].x = 
(input[tid].x + input[tid].y + input[tid].z) / 3; output[tid].z = output[tid].y = output[tid].x; } void 
Labwork::labwork4_GPU() { int pixelCount = inputImage->width * inputImage->height;
         outputImage = static_cast<char *>(malloc(pixelCount * 3)); // Calculate number of pixels // Allocate 
CUDA memory uchar3 *devGray;
         uchar3 *devInput;
         cudaMalloc(&devInput, pixelCount * sizeof(uchar3));
        cudaMalloc(&devGray, pixelCount * sizeof(uchar3));
 // Copy CUDA Memory from CPU to GPU //cudaMemcpy(devGray, outputImage, pixelCount * sizeof(uchar3), 
cudaMemcpyHostToDevice);
 cudaMemcpy(devInput, inputImage->buffer, pixelCount * sizeof(uchar3),cudaMemcpyHostToDevice);
 // Processing
  //rgb2grayCUDA<<<dimGrid, dimBlock>>>(devInput,devGray,regionSize); dim3 blockSize = dim3(16,8);
        dim3 gridSize = dim3(inputImage->width/blockSize.x,inputImage->height/blockSize.y);
        grayscale2<<<gridSize, blockSize>>>(devInput, devGray);
        //int blockSize =128; // int numBlock = pixelCount / blockSize; // grayscale<<<numBlock, 
blockSize>>>(devInput, devGray); // Copy CUDA Memory from GPU to CPU //cudaMencpy()
        cudaMemcpy(outputImage, devGray , pixelCount*sizeof(uchar3),cudaMemcpyDeviceToHost); 
//cudaMemcpy(inputImage->buffer, devInput, pixelCount * sizeof(uchar3),cudaMemcpyHostToDevice); // Cleaning 
cudaFree(devInput);
         cudaFree(devGray);
}             
We calculate the time difference student5@ictserver2:/storage/student5/advancedhpc2019/labwork/build$ ./labwork 
3 ../data/eiffel.jpg USTH ICT Master 2018, Advanced Programming for HPC. Warming up... Starting labwork 3 
labwork 3 ellapsed 181.7ms student5@ictserver2:/storage/student5/advancedhpc2019/labwork/build$ ./labwork 4 
../data/eiffel.jpg for 16/8 configuration USTH ICT Master 2018, Advanced Programming for HPC. Warming up... 
Starting labwork 4 labwork 4 ellapsed 177.7ms We can see that we gain 4ms for 16/16 configuration we have USTH 
ICT Master 2018, Advanced Programming for HPC. Warming up... Starting labwork 4 labwork 4 ellapsed 173.8ms so 
it's faster for 8/8 configuration USTH ICT Master 2018, Advanced Programming for HPC. Warming up... Starting 
labwork 4 labwork 4 ellapsed 179.7ms so it's slower for 32/16 configuration we have USTH ICT Master 2018, 
Advanced Programming for HPC. Warming up... Starting labwork 4 labwork 4 ellapsed 178.8ms So the 16/16 is the 
fastest one \end{verbatim} \end{document}
